\section{Message Format}
\subsection{Header Format}
Der \ac{CoAP}-Header, wie in Abbildung~\ref{fig:messageformat} zu sehen, hat damit eine Mindestlänge von 4 Byte.
Je nach Nachricht, folgt danach der Token und die Optionen, die für das Bearbeiten der Anfrage oder
das Verstehen der Antwort wichtig sind.
Vorallem bei Response-Nachrichten gibt es zusätzlich am Ende der Nachricht noch einen Payload, in
dem beispielsweise die abgefragte Repräsentation einer Ressource enthalten ist.
\begin{figure}[htbp]
    \centering
    \begin{minipage}{0.95\textwidth}
    \lstinputlisting[]{sources/messageFormat.txt}
    \end{minipage}
\begin{center}
\begin{tabular}{l|c|c|l}
    Name & Länge & Wertebereich & Beschreibung \\
    \hline
    Version     & 2 - Bit   & 1     & CoAP Version \\
    \hline
    Type        & 2 - Bit   & 0~-~3 & Signalisiert den Nachrichtentyp \\
                &           &   0   & Confirmable \\
                &           &   1   & Non~-~Confirmable \\
                &           &   2   & Acknowledgement\\
                &           &   3   & Reset \\
    \hline
    Token Length& 4 - Bit   & 0 - 8 & Gibt die variable Länge des Tokens an \\
                &           & 0 - 8 & Länge des Tokens \\
                &           & 0 -15 & reserviert \\
    \hline
    Code        & 8 - Bit   &0 - 31,64 - 191    & Zeigt, ob es sich um eine \\
                &           &   0    & Empty Message,\\
                &           & 1 - 31 & Request Message (siehe Abb. \ref{table:requestCodes}) oder\\
                &           &64 - 191& Response Message (siehe Abb. \ref{table:responseCodes}) handelt\\
    \hline
    Message ID  & 16 - Bit  &        & unigned integer in Network Byte order\\
                &           &        & dient dem Request/Response Matching \\ 
\end{tabular}
\end{center}
\caption{Message Format}
\label{fig:messageformat}
\end{figure}
\subsubsection{Tokenlänge statt Option Count}
Mit dem draft-13[\cite{draft-ietf-core-coap-13}] von CoAP, welches am 6. Dezember 2012 erschien, hat der Token nun seinen Weg direkt in
den Header einer \ac{CoAP}-Nachricht gefunden.
Vorher wurde der Token wie eine normale Option behandelt.
Statt der jetzigen Token-Länge im Header wurde damals die Anzahl der Optionen
übergeben.
Doch da sich herausstellte, dass allein die Message-ID oft nicht ausreicht, um
eine Response-Nachricht ihrer Request-Nachricht zuordnen zu können, oder man den
Server damit nicht mehr die Wahl lassen konnte seine Antwort piggybacked (siehe
Abschnitt~\ref{piggybackedResponse} \nameref{piggybackedResponse}) oder seperat
(siehe Abschnitt~\ref{seperateResponse} \nameref{seperateResponse}) abzusenden.
Bei der Verwendung von mehr als 14 Optionen musste
damals ein zusätzlicher End-Of-Option-Marker benutzt werden, da im
Option-Count-Feld mit der Länge 4 Bit nicht mehr angegeben werden konnte.
Ein Option-Count von 15 bedeutete dementsprechend, dass der oben erwähnte Marker
als Trenner zwischen Optionen und Payload diente.
Statt dies nur in Ausnahmen so zu handhaben, ist es in der aktuellen Version
Standard.
Dafür müssen die Anzahl der Optionen nicht mehr angegeben werden und der, meist
ohnehin verwendete, Token beschreibt in diesem Feld nun seine Länge.
Bei einer Token Länge von 0 ist der Token mit dem Wert 0 anzusehen.
Um nun allerdings Header, Optionen und Payload einer \ac{CoAP}-Nachricht
auseinander halten zu können, verwendet man den oben bereits erwähnten
End-Of-Option-Marker. Allerdings nennt er sich nun Payloadmarker, welcher aus
einem Byte mit Einsen besteht.
Bei Optionen ist deshalb nun darauf zu achten, dass das erste Byte eben genau
nicht aus einem Byte voller Einsen besteht.

\subsubsection{Offengelassene Erweiterungsmöglichkeiten}
Wie leicht zu erkennen ist, ist im Header noch viel Platz für spätere Ideen im \ac{CoAP}-Protokoll.
Das Token-Längenfeld darf gerade mal acht der 15 möglichen Werte annehmen.
Dies reicht für die Tokenlänge auch aus, doch finden die restlichen Werte in dem Feld noch keine
Verwendung.
Ebenso gibt es noch viel Spielraum im Code-Feld.
Dort werden nur 159 der 256 möglichen Werte ausgenutzt.
Auch hier ist man schon sehr großzügig gewesen und hat sehr viele Fehlercodes aus dem
HTTP-Protokoll (nach \cite{rfc2616}) übernommen.
Es ist also durchaus möglich, dass bis zum Übergang des drafts in einen \ac{RFC}, noch weitere
Anpassungen, Alternativen oder Optimierungen am Header vorgenommen werden.
\newpage
\subsection{Option Format}
\label{optionFormat}
In \ac{CoAP} können keine bis zu theoretisch unendlich vielen Optionen zwischen Header und 
Payload eingehängt werden.
Alle Optionen sind, wie in Abbildung~\ref{fig:optionformat} zu sehen, immer nach dem Schema \ac{TLV} aufgebaut.
Als Besonderheit ist hier alledings zu beachten, dass der Typ nicht explizit
angegeben wird (siehe Abschnitt~\ref{optiondelta} \nameref{optiondelta}).
Bei den Optionen wurde vorallem darauf Wert gelegt, sie sehr variabel gestalten zu können.
Optionen, die beispielsweise keinen Wert benötigen, wie die "`If-None-Match"' Option
(Optionsnummer~5), wird so nur 1~Byte groß.
Ebenso ist es aber auch möglich sehr große Optionswerte zu übetragen.
\begin{figure}[htbp]
    \centering
    \begin{minipage}{0.75\textwidth}
    \lstinputlisting[]{sources/optionFormat.txt}
    \end{minipage}
\begin{center}
\begin{tabular}{l|c|c|l}
    Name & Länge & Wert & Beschreibung \\
    \hline
    Option Delta & 4 - Bit &  & Gibt die Optionsnummerndifferenz\\
                            &&&zur vorherigen Option oder\\
                            &&&von 0 bei der ersten Option an\\
                 &       &0 - 12& Optionsnummerndifferenz\\
                 &       &13& 8 - Bit Options Delta folgt\\
                 &       &14& 16 - Bit Options Delta folgt\\
                 &       &15& Reserviert für Payload Marker\\
    \hline
    Options Length & 4 - Bit &  & Gibt die Länge des\\ 
                            &&&Optionswertes an\\
                 &       &0 - 12& Länge in Bytes\\
                 &       &13& 8 - Bit Options Länge folgt\\
                 &       &14& 16 - Bit Options Länge folgt\\
                 &       &15& Reserviert\\
    \hline
    optionales Delta& 8 - Bit  &Delta - 13& Für Werte zwischen 13 - 268 \\
    \hline
    optionales Delta& 16 - Bit &Delta - 269& Für Werte zwischen 269 - 65804\\    
    \hline
    optionale Länge& 8 - Bit  &Länge - 13& Für Werte zwischen 13 - 268  \\
    \hline
    optionale Länge& 16 - Bit &Länge - 269& Für Werte zwischen 269 - 65804\\
    \hline
    Option Value &Option Length&& Wert der Option
\end{tabular}
\end{center}
\caption{Option Format}
\label{fig:optionformat}
\end{figure}
\subsubsection{Optionsnummern Differenz}
\label{optiondelta}
Da sich einige Optionen wiederholen können, wäre es nicht sinnvoll immer die komplette Optionsnummer
zu übertragen, wenn diese sehr groß ist.
Aus diesem Grund wird bei \ac{CoAP} immer die Differenz zur vorherigen Optionsnummer angegeben.
Das bedeutet, dass eine Folge von gleichen Optionen mit einer großen Optionsnummer nur ein Delta von 0
angeben müssen, anstatt ihrer Optionsnummer.
Dies hat als Nebeneffekt zur Folge, das die Optionen sortiert nach ihrer Optionsnummer aufgelistet
werden, da Rückwärtssprünge nicht möglich sind.

Ein Problem hat das Delta-Prinzip allerdings, denn werden große Sprünge gemacht, kann das 4-Bit
große Feld unter Umständen nicht ausreichen.
In frühen \ac{CoAP}-Versionen wurden deshalb sogenannte Fencepost-Optionen eingefügt.
Diese hatten als Optionsnummer immer ein Vielfaches von 14.
Wollte man beispielsweise als erstes eine Option mit der Nummer 30 an seine Nachricht hängen, so
musste man erst 2 Optionen mit dem Delta 14 angeben und erst danach die Option mit einem Delta von
2.
Es wurden also 2 Bytes für die Fencepost Optionen benötigt um diesen Sprung durchführen zu können.
Ab CoAP-12[\cite{draft-ietf-core-coap-12}] gibt es nun keine Fencepost Optionen mehr.
Dafür gibt es jetzt sogenannte "`Extended"'-Felder.
Werden große Sprünge benötigt oder ist der Wert der Option sehr lang, so gibt es extra Felder um
diesen großen Wert auszudrücken.
Bei unserer Option mit der Nummer 30 beisielsweise, würden wir als Option-Delta die 13 eintragen und
haben uns so ein 8-Bit großes Extended-Delta-Feld , wie oben beschrieben, reserviert.
In diesem Feld brauchen wir nun nur noch eine 17 (= 30 - 13) eintragen.
Anstatt wie bei der Fencepost-Methode 2 Byte für diesen großen Sprung zu bezahlen, zahlen wir mit
der Extended-Feld-Methode nur noch 1 Byte.
Bei größeren Sprüngen, ist die Ersparnis noch größer.

\subsubsection{Optionswerte}
In \ac{CoAP}-Optionen gibt es nur vier verschiedene Typen, die auftreten können.
Eine Option hat auch immer den gleichen Datentyp, wie in \cite{draft-ietf-core-coap-13} beschrieben(siehe Abbildung~\ref{table:coapOptions}).
Es muss also nicht jedes mal in der Option beschrieben werden, in welchem Format der Optionswert übertragen wurde.
\begin{description}
\item[empty] Kein Optionswert (0-Byte Sequenz von Bytes)
\item[opaque] eine Sequenz von Bytes, die nicht interpretiert werden muss
\item[uint] eine nicht-negative Zahl in Network Byte Order
\item[string] ein Unicode String kodiert in \ac{UTF-8} (nach \cite{rfc3629}) als Net-Unicode form (nach \cite{rfc5198})
\end{description} 