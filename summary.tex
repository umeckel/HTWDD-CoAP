\section{Zusammenfassung}
\ac{CoAP} implementiert eine \ac{REST}ful Architektur mit möglichst wenig Overhead,
Was es vorallem im Einsatz mit Sensornetzen interessant macht: Zum einen ist es
möglich, Sensorwerte abzufragen, zum anderen bietet es auch die Möglichkeit, Werte
an den Endpoints zu ändern. Durch das Verwenden von \ac{UDP} als Transportschicht,
ist zwar die Zuverlässigkeit nicht mehr gesichert, doch spart man einiges an
Overhead gegenüber \ac{TCP}. Ist man daran interessiert, dennoch
Zuverlässigkeitsmechanismen zu benutzen, so bietet auch dort \ac{CoAP}
Möglichkeiten. Weiterhin bietet das Verwenden einer \ac{REST}-Architektur leichte
Anbindung an andere \ac{REST}-Netzwerke wie z.B. \ac{HTTP}. Ein Mapping von \ac{CoAP}
auf \ac{HTTP} wurde in diesem Dokument nicht beschrieben, doch ist dies auch bereits
in \cite{draft-ietf-core-coap-13} beschrieben. Durch Proxys und den Einsatz von den
oben beschriebenen Proxy-*-Optionen, ist es möglich auch aus dem \ac{CoAP} Netzwerk
heraus oder hinein Anfragen zu stellen.

Vorallem wenn man die Mailingliste von \ac{CoAP} verfolgt, stellt man fest, dass
es zwar doch noch an einigen Stellen zu nicht-vorhergesehenen Fehlern kommt, doch
durch die aktive Arbeit mehrerer Leute wird dies ausführlich diskutiert und
Änderungen werden im Draft festgehalten. Trotz dieser auftretenen Probleme ist
\ac{CoAP} in einem Zustand, in dem es implementiert und bearbeitet werden kann.
Bei der Implementierung allerdings scheitert es derzeit. Nur für wenige Sprachen
sind derzeit Implementierungen einzelner Leute im Internet zu finden. Immerhin
für die Programmiersprache C gibt es eine aktive
Gruppe\footnote{\url{http://libcoap.sourceforge.net/}}, welche versucht ihre
Implementierung auf dem aktuellen Stand zu halten.

Um \ac{CoAP} hinsichtlich Effizienz zu beurteilen fehlt es dadurch leider an
Testberichten. Doch die Aufmerksamkeit steigt und nachdem nun die Entwicklung
einen Stand erreicht hat, indem es sich lohnt, eine Bibliothek zu programmieren,
wird es schon möglich sein, bessere Aussagen zu treffen. Aus theorethischer Sicht
macht CoAP definitiv Sinn und hat gute Aussichten für die Zukunft, vorallem im
Bereich Sensornetzen und anderen Low-Power \ac{M2M}-Netzwerken.

 