\documentclass[a4paper, oneside, 12pt]{scrartcl}

\usepackage{fancyhdr}
\pagestyle{headings}

\usepackage[ngerman]{babel}
\usepackage[utf8]{inputenc}
\usepackage[T1]{fontenc}

\usepackage{hyperref}
\usepackage[all]{hypcap} %um auf figures, statt auf die Labels zu verweisen

\usepackage{url}
\usepackage{graphicx}

\setcounter{tocdepth}{3} %um subsubsections mit in den ToC einzugliedern

\usepackage{listingsutf8} \lstset{captionpos=b, basicstyle=\small, extendedchars=true}

\lstset{literate=%
{Ö}{{\"O}}1
{Ä}{{\"A}}1
{Ü}{{\"U}}1
{ß}{{\ss}}2
{ü}{{\"u}}1
{ä}{{\"a}}1
{ö}{{\"o}}1
{-}{{\--}}1
}

\usepackage[nohyperlinks]{acronym}
\makeindex

\title{CoAP - Constrained Application Protocol}
\author{Martin~Dönicke, Ulrich~Meckel}
\date{\today}

\begin{document}
%\vfill{}
%\maketitle
\begin{titlepage}
\begin{center}
\vspace*{380pt}
\huge{\textbf{CoAP\\Constrained Application Protocol}}\\
\vspace{0.75cm}
\large{Lehrveranstaltung Sensornetze}\\
\large{der Fakultät Informatik/Mathematik}\\
\vspace{1.5cm}
\end{center}
\vfill
Eingereicht von: Martin~Dönicke, Ulrich Meckel\\
Eingereicht am: 15.~Februar~2013\\
%Betreuender Hochschullehrer: Prof.~Dr.-Ing.~Robert~Baumgartl\\
\end{titlepage}

\thispagestyle{empty}
\newpage
\tableofcontents
\section*{Abkürzungsverzeichnis}
\thispagestyle{plain}
\addcontentsline{toc}{section}{Abkürzungsverzeichnis}
\thispagestyle{plain}
\begin{acronym}
\acro{6LoWPAN}{IPv6 over Low power Wireless Personal Area Network}
\acro{6LBR}{6LoWPAN Border Router}
\acro{ACK}{Acknowledgement}
\acro{CoAP}{Constrained Application Protocol}
\acro{CON}{Confirmable}
\acro{CoRE}{Constrained \acs{REST}ful Environments}
\acro{DNS}{Domain Name System}
\acro{DTLS}{Datagram Transport Layer Security}
\acro{FQDN}{Fully Qualified Domain Name}
\acro{HTTP}{Hypertext Transfer Protocol}
\acro{GPRS}{General Packet Radio Service}
\acro{IETF}{Internet Engineering Task Force}
\acro{IP}{Internet Protocol}
\acro{IPSec}{\ac{IP} Security}
\acro{M2M}{machine-to-machine}
\acro{MLD}{Multicast Listener Discovery}
\acro{MPL}{Multicast Protocol for Low power and Lossy Networks}
\acro{NON}{Non-confirmable}
\acro{NoSec}{No Security}
\acro{ODP}{Observe-Design-Pattern}
\acro{PHY}{Physical Layer}
\acro{RAM}{Random-Access-Memory}
\acro{REST}{Representational State Transfer}
\acro{RFC}{Request for Comments}
\acro{ROM}{Read-Only-Memory}
\acro{RPL}{Routing Protocol for Low power and Lossy Networks}
\acro{RST}{Reset}
\acro{SMS}{Short Messaging Service}
\acro{TCP}{Transmission Control Protocol}
\acro{TLV}{Type-Length-Value}
\acro{UCS}{Universal Character Set}
\acro{USSD}{Unstructured Supplementary Service Data}
\acro{UDP}{User Datagram Protocol}
\acro{UTF-8}{8-Bit \acs{UCS} Transformation Format}
\acro{URI}{Uniform Resource Identifier}
\end{acronym}
\thispagestyle{plain}
\section*{Glossar}
\thispagestyle{plain}
\addcontentsline{toc}{section}{Glossar}
\begin{description}
  \item[Endpoint] Einheit im \acs{CoAP} Netzwerk, aber nicht zwingend ein Endknoten
  \vspace{18pt}
  \item[Sender] Der Ursprungs-Endpoint einer Nachricht
  \item[Recipient] Der Ziel-Endpoint einer Nachricht
  \vspace{18pt}
  \item[Client] Der Ursprungs-Endpoint einer Request-Nachricht und Ziel-Endpoint einer Response-Nachricht
  \item[Server] Der Ziel-Endpoint einer Request-Nachricht und Ursprungs-Endpoint einer Response-Nachricht
  \vspace{18pt} 
  \item[Origin Server] Ein Server auf dem eine gewisse Ressource angelegt ist
  \item[Intermediary] Ein Endpoint, der sowohl als Server, als auch als Client fungiert (bspw. ein Proxy)
  \vspace{18pt}
  \item[Forward-Proxy] Dient dem Weiterleiten von Nachrichten
  \item[Reverse-Proxy] Bietet fremde Ressourcen als seine eigenen an
  \item[Cross-Proxy] Ermöglicht Kommunikation mit \acs{CoAP}-Knoten über ein anderes Protokoll
\end{description}
\section{Einleitung}
Das \ac{CoAP} ist ein für den Einsatz in Sensornetzen konzipiertes Transferprotokoll.
Spezielle Randbedingungen in solchen Netzen sind wenig \ac{RAM} und \ac{ROM} der einzelnen
Kommunikationsteilnehmer, welche z.B. durch 8-Bit Mikrocontroller realisiert werden.
Weiterhin bieten Sensornetze im Vergleich zu üblichen Computernetzen nur geringe Datenraten und
haben möglicherweise hohe Paketfehlerraten.

Ein mögliches Netz in dem \ac{CoAP} zum Einsatz kommen könnte, wäre
beispielsweise ein \ac{6LoWPAN}-Netz, in dem \ac{CoAP}-Nachrichten via \ac{UDP}
versendet werden.
Außer UDP in der Transportschicht, werden jedoch keine weiteren unterliegenden
Schichten für \ac{CoAP} spezifiziert.
Es wird lediglich versucht, den Overhead der einzelnen Nachrichten gering zu
halten.
Zudem hat \ac{CoAP} eine \ac{REST}-Architektur (siehe \cite{Fielding:2000:ASD:932295}), weshalb die Last auf
speicherarmen Knoten nicht so hoch ist, da keine Status gehalten werden müssen.
Wie genau das versucht wird zu erreichen, wird später in diesem Dokument
erläutert.
Die Entwicklung von \ac{CoAP} wird von der \ac{CoRE}-Gruppe vorangetrieben. Der
erste Entwurf wurde im Jahr 2010 veröffentlicht.
Zum Zeitpunkt der Erstellung dieses Dokuments war Version~13 (siehe \cite{draft-ietf-core-coap-13}) aktuell. Es wird
versucht auf einige Änderung einzugehen, die während der Entwicklung entstanden
sind.

\subsection{Features}
In \cite{draft-ietf-core-coap-13} wird \ac{CoAP} mit folgenden Features beworben:
\begin{itemize}
    \item Constrained web protocol fulfilling \ac{M2M} requirements.
    \item \ac{UDP} binding with optional reliability supporting unicast and multicast requests.
    \item Asynchronous message exchanges.
    \item Low header overhead and parsing complexity.
    \item \ac{URI} and Content-type support.
    \item Simple proxy and caching capabilities.
    \item A stateless \ac{HTTP} mapping, allowing proxies to be built providing access to \ac{CoAP}
    resources via \ac{HTTP} in a uniform way or for \ac{HTTP} simple interfaces to be realized
    alternatively over \ac{CoAP}.
    \item Security binding to \ac{DTLS}.
\end{itemize}

\subsection{Abgrenzung und Ziel}
Ziel von \ac{CoAP} ist es nicht, ein komprimiertes \ac{HTTP} umzusetzen, sondern viel mehr einen
Teil von \ac{REST}, wie es in \ac{HTTP} zum Einsatz kommt, bereitzustellen.
Insbesondere Optimierungen für eine \ac{M2M}-Kommunikation wie beispielsweise eingebaute Discovery-
Mechanismen, Unterstützung für Multicasts und asynchronen Nachrichtenaustausch sollen durch
\ac{CoAP} bereitgestellt werden.
\section{Message Format}
\subsection{Header Format}
Der \ac{CoAP}-Header, wie in Abbildung~\ref{fig:messageformat} zu sehen, hat damit eine Mindestlänge von 4 Byte.
Je nach Nachricht, folgt danach der Token und die Optionen, die für das Bearbeiten der Anfrage oder
das Verstehen der Antwort wichtig sind.
Vorallem bei Response-Nachrichten gibt es zusätzlich am Ende der Nachricht noch einen Payload, in
dem beispielsweise die abgefragte Repräsentation einer Ressource enthalten ist.
\begin{figure}[htbp]
    \centering
    \begin{minipage}{0.95\textwidth}
    \lstinputlisting[]{sources/messageFormat.txt}
    \end{minipage}
\begin{center}
\begin{tabular}{l|c|c|l}
    Name & Länge & Wertebereich & Beschreibung \\
    \hline
    Version     & 2 - Bit   & 1     & CoAP Version \\
    \hline
    Type        & 2 - Bit   & 0~-~3 & Signalisiert den Nachrichtentyp \\
                &           &   0   & Confirmable \\
                &           &   1   & Non~-~Confirmable \\
                &           &   2   & Acknowledgement\\
                &           &   3   & Reset \\
    \hline
    Token Length& 4 - Bit   & 0 - 8 & Gibt die variable Länge des Tokens an \\
                &           & 0 - 8 & Länge des Tokens \\
                &           & 0 -15 & reserviert \\
    \hline
    Code        & 8 - Bit   &0 - 31,64 - 191    & Zeigt, ob es sich um eine \\
                &           &   0    & Empty Message,\\
                &           & 1 - 31 & Request Message (siehe Abb. \ref{table:requestCodes}) oder\\
                &           &64 - 191& Response Message (siehe Abb. \ref{table:responseCodes}) handelt\\
    \hline
    Message ID  & 16 - Bit  &        & unigned integer in Network Byte order\\
                &           &        & dient dem Request/Response Matching \\ 
\end{tabular}
\end{center}
\caption{Message Format}
\label{fig:messageformat}
\end{figure}
\subsubsection{Tokenlänge statt Option Count}
Mit dem draft-13[\cite{draft-ietf-core-coap-13}] von CoAP, welches am 6. Dezember 2012 erschien, hat der Token nun seinen Weg direkt in
den Header einer \ac{CoAP}-Nachricht gefunden.
Vorher wurde der Token wie eine normale Option behandelt.
Statt der jetzigen Token-Länge im Header wurde damals die Anzahl der Optionen
übergeben.
Doch da sich herausstellte, dass allein die Message-ID oft nicht ausreicht, um
eine Response-Nachricht ihrer Request-Nachricht zuordnen zu können, oder man den
Server damit nicht mehr die Wahl lassen konnte seine Antwort piggybacked (siehe
Abschnitt~\ref{piggybackedResponse} \nameref{piggybackedResponse}) oder seperat
(siehe Abschnitt~\ref{seperateResponse} \nameref{seperateResponse}) abzusenden.
Bei der Verwendung von mehr als 14 Optionen musste
damals ein zusätzlicher End-Of-Option-Marker benutzt werden, da im
Option-Count-Feld mit der Länge 4 Bit nicht mehr angegeben werden konnte.
Ein Option-Count von 15 bedeutete dementsprechend, dass der oben erwähnte Marker
als Trenner zwischen Optionen und Payload diente.
Statt dies nur in Ausnahmen so zu handhaben, ist es in der aktuellen Version
Standard.
Dafür müssen die Anzahl der Optionen nicht mehr angegeben werden und der, meist
ohnehin verwendete, Token beschreibt in diesem Feld nun seine Länge.
Bei einer Token Länge von 0 ist der Token mit dem Wert 0 anzusehen.
Um nun allerdings Header, Optionen und Payload einer \ac{CoAP}-Nachricht
auseinander halten zu können, verwendet man den oben bereits erwähnten
End-Of-Option-Marker. Allerdings nennt er sich nun Payloadmarker, welcher aus
einem Byte mit Einsen besteht.
Bei Optionen ist deshalb nun darauf zu achten, dass das erste Byte eben genau
nicht aus einem Byte voller Einsen besteht.

\subsubsection{Offengelassene Erweiterungsmöglichkeiten}
Wie leicht zu erkennen ist, ist im Header noch viel Platz für spätere Ideen im \ac{CoAP}-Protokoll.
Das Token-Längenfeld darf gerade mal acht der 15 möglichen Werte annehmen.
Dies reicht für die Tokenlänge auch aus, doch finden die restlichen Werte in dem Feld noch keine
Verwendung.
Ebenso gibt es noch viel Spielraum im Code-Feld.
Dort werden nur 159 der 256 möglichen Werte ausgenutzt.
Auch hier ist man schon sehr großzügig gewesen und hat sehr viele Fehlercodes aus dem
HTTP-Protokoll (nach \cite{rfc2616}) übernommen.
Es ist also durchaus möglich, dass bis zum Übergang des drafts in einen \ac{RFC}, noch weitere
Anpassungen, Alternativen oder Optimierungen am Header vorgenommen werden.
\newpage
\subsection{Option Format}
\label{optionFormat}
In \ac{CoAP} können keine bis zu theoretisch unendlich vielen Optionen zwischen Header und 
Payload eingehängt werden.
Alle Optionen sind, wie in Abbildung~\ref{fig:optionformat} zu sehen, immer nach dem Schema \ac{TLV} aufgebaut.
Als Besonderheit ist hier alledings zu beachten, dass der Typ nicht explizit
angegeben wird (siehe Abschnitt~\ref{optiondelta} \nameref{optiondelta}).
Bei den Optionen wurde vorallem darauf Wert gelegt, sie sehr variabel gestalten zu können.
Optionen, die beispielsweise keinen Wert benötigen, wie die "`If-None-Match"' Option
(Optionsnummer~5), wird so nur 1~Byte groß.
Ebenso ist es aber auch möglich sehr große Optionswerte zu übetragen.
\begin{figure}[htbp]
    \centering
    \begin{minipage}{0.75\textwidth}
    \lstinputlisting[]{sources/optionFormat.txt}
    \end{minipage}
\begin{center}
\begin{tabular}{l|c|c|l}
    Name & Länge & Wert & Beschreibung \\
    \hline
    Option Delta & 4 - Bit &  & Gibt die Optionsnummerndifferenz\\
                            &&&zur vorherigen Option oder\\
                            &&&von 0 bei der ersten Option an\\
                 &       &0 - 12& Optionsnummerndifferenz\\
                 &       &13& 8 - Bit Options Delta folgt\\
                 &       &14& 16 - Bit Options Delta folgt\\
                 &       &15& Reserviert für Payload Marker\\
    \hline
    Options Length & 4 - Bit &  & Gibt die Länge des\\ 
                            &&&Optionswertes an\\
                 &       &0 - 12& Länge in Bytes\\
                 &       &13& 8 - Bit Options Länge folgt\\
                 &       &14& 16 - Bit Options Länge folgt\\
                 &       &15& Reserviert\\
    \hline
    optionales Delta& 8 - Bit  &Delta - 13& Für Werte zwischen 13 - 268 \\
    \hline
    optionales Delta& 16 - Bit &Delta - 269& Für Werte zwischen 269 - 65804\\    
    \hline
    optionale Länge& 8 - Bit  &Länge - 13& Für Werte zwischen 13 - 268  \\
    \hline
    optionale Länge& 16 - Bit &Länge - 269& Für Werte zwischen 269 - 65804\\
    \hline
    Option Value &Option Length&& Wert der Option
\end{tabular}
\end{center}
\caption{Option Format}
\label{fig:optionformat}
\end{figure}
\subsubsection{Optionsnummern Differenz}
\label{optiondelta}
Da sich einige Optionen wiederholen können, wäre es nicht sinnvoll immer die komplette Optionsnummer
zu übertragen, wenn diese sehr groß ist.
Aus diesem Grund wird bei \ac{CoAP} immer die Differenz zur vorherigen Optionsnummer angegeben.
Das bedeutet, dass eine Folge von gleichen Optionen mit einer großen Optionsnummer nur ein Delta von 0
angeben müssen, anstatt ihrer Optionsnummer.
Dies hat als Nebeneffekt zur Folge, das die Optionen sortiert nach ihrer Optionsnummer aufgelistet
werden, da Rückwärtssprünge nicht möglich sind.

Ein Problem hat das Delta-Prinzip allerdings, denn werden große Sprünge gemacht, kann das 4-Bit
große Feld unter Umständen nicht ausreichen.
In frühen \ac{CoAP}-Versionen wurden deshalb sogenannte Fencepost-Optionen eingefügt.
Diese hatten als Optionsnummer immer ein Vielfaches von 14.
Wollte man beispielsweise als erstes eine Option mit der Nummer 30 an seine Nachricht hängen, so
musste man erst 2 Optionen mit dem Delta 14 angeben und erst danach die Option mit einem Delta von
2.
Es wurden also 2 Bytes für die Fencepost Optionen benötigt um diesen Sprung durchführen zu können.
Ab CoAP-12[\cite{draft-ietf-core-coap-12}] gibt es nun keine Fencepost Optionen mehr.
Dafür gibt es jetzt sogenannte "`Extended"'-Felder.
Werden große Sprünge benötigt oder ist der Wert der Option sehr lang, so gibt es extra Felder um
diesen großen Wert auszudrücken.
Bei unserer Option mit der Nummer 30 beisielsweise, würden wir als Option-Delta die 13 eintragen und
haben uns so ein 8-Bit großes Extended-Delta-Feld , wie oben beschrieben, reserviert.
In diesem Feld brauchen wir nun nur noch eine 17 (= 30 - 13) eintragen.
Anstatt wie bei der Fencepost-Methode 2 Byte für diesen großen Sprung zu bezahlen, zahlen wir mit
der Extended-Feld-Methode nur noch 1 Byte.
Bei größeren Sprüngen, ist die Ersparnis noch größer.

\subsubsection{Optionswerte}
In \ac{CoAP}-Optionen gibt es nur vier verschiedene Typen, die auftreten können.
Eine Option hat auch immer den gleichen Datentyp, wie in \cite{draft-ietf-core-coap-13} beschrieben(siehe Abbildung~\ref{table:coapOptions}).
Es muss also nicht jedes mal in der Option beschrieben werden, in welchem Format der Optionswert übertragen wurde.
\begin{description}
\item[empty] Kein Optionswert (0-Byte Sequenz von Bytes)
\item[opaque] eine Sequenz von Bytes, die nicht interpretiert werden muss
\item[uint] eine nicht-negative Zahl in Network Byte Order
\item[string] ein Unicode String kodiert in \ac{UTF-8} (nach \cite{rfc3629}) als Net-Unicode form (nach \cite{rfc5198})
\end{description} 
\section{Optionen}
Im Abschnitt~\ref{optionFormat} \nameref{optionFormat} ist bereits der Aufbau einer Option erläutert.
In diesem Kapitel soll einmal genau der Aufbau der Optionsnummer betrachtet werden und ebenso
welche verschiedenen Optionen CoAP zur Verfügung stellt und wozu diese notwendig sind.
\begin{figure}[htbp]
    \centering
    \begin{minipage}{.97\textwidth}
    \lstinputlisting[basicstyle=\footnotesize]{sources/tableOptions.txt}
    \caption{CoAP Optionen}
    \end{minipage}
    \label{table:coapOptions}
\end{figure}
\subsection{Eigenschaften}
\begin{figure}[htbp]
    \centering
    \begin{minipage}{.5\textwidth}
    \lstinputlisting[basicstyle=\footnotesize]{sources/optionNumberMask.txt}
    \caption{Optionsnummernmaske}
    \end{minipage}
    \label{table:optionnumbermask}
\end{figure}
\subsubsection{Critical / Elective}
Ist eine Option als kritisch markiert, muss diese verstanden werden, andernfalls führt sie zu
folgenden drei Fehlerfällen:
\begin{description}
  \item [\ac{CON}-Request] 
  Kann die Option bei einer \ac{CON}-Request-Nachricht nicht verstanden
werden, muss sie mit dem Fehlercode \verb!4.02! (Bad Option) beantwortet werden.
Zusätzlich sollte im Payload einer solchen Antwort eine kurze,
menschlich lesbare, \ac{UTF-8} kodierte Nachricht stehen, die den Fehler beschreibt.
Beispielsweise bei welcher Option der Fehler überhaupt auftrat und was nicht
verstanden wurde oder man beispielsweise die Option überhaupt nicht kennt.
  \item [\ac{CON}-Response]
  Kann bei einer \ac{CON}-Response oder einer durch
Piggy-backed-Response (siehe Abschnitt~\ref{piggybackedResponse} \nameref{piggybackedResponse}) 
erhaltenen Nachricht eine kritische Option nicht verstanden werden, muss
diese mit einer \ac{RST}-Nachricht beantwortet werden und die Response selbst aber ignoriert
werden.
  \item [\ac{NON}-Nachricht]
  Nachrichten mit einer oder mehreren nichtverstandenen kritischen Optionen müssen ignoriert
  werden.
  Der Empfänger einer solchen fehlerhaften oder nicht verstandenen Option kann diese Nachricht auch
  durch eine \ac{RST}-Nachricht beantworten, muss sie aber dennoch ignorieren. 
\end{description}
Kann die Option nicht verstanden werden und ist aber nicht kritisch, so wird die Option ignoriert,
die Nachricht allerdings nicht.

Ob eine Option kritisch ist oder nicht, entscheidet sich durch das niedrigstwertige Bit der
Optionsnummer, wie in Abbildung~\ref{table:optionnumbermask} zu sehen. Ist es gesetzt, so ist die Option kritisch, sonst nicht. Damit sind alle Optionen mit
einer ungeraden Optionsnummer kritisch.

\subsubsection{Proxy Unsafe/Safe}
Proxys müssen nicht alle Optionen verstehen können. Aus diesem Grund, ist es für diese Gruppe von
Endpoints nicht relevant, ob die Optionen kritisch sind oder nicht sind. An dieser Stelle wird
zwischen Safe-to-Forward oder Unsafe-to-Forward unterschieden. Versteht ein Proxy also eine Option
nicht, die mit Unsafe-to-Forward markiert ist, muss er diese mit einem \verb!4.02! (Bad Option)
Fehlercode beantworten.
Safe-to-Forward Optionen werden von Proxys nochmals unterschieden. Nämlich ob sie zum Cache-Key gehören oder
nicht.
Wozu diese zusätzliche Unterscheidung dient, wird bei der Erklärung der ETag-Option (siehe Abschnitt~\ref{option:etag} \nameref{option:etag})
verdeutlicht.
Das zweit-niedrigstwertige Bit der Optionsnummer entscheidet darüber, ob die Option Safe-to-Forward
ist oder nicht, vergleiche Abbildung~\ref{table:optionnumbermask}. Ist es gesetzt, so ist die Option Unsafe-to-Forward. Ist es nicht gesetzt, so
unterscheiden die 3 nächsthöheren Bits darüber, ob die Option zum Cache-Key gehört oder nicht. Ist
eines der 3 Bits nicht gesetzt, so gehört die Safe-to-Forward Option zum Cache-Key.

\subsubsection{Repeatable}
Manche Optionen können in \ac{CoAP} mehrfach vorkommen. Dies ist zum Beispiel wichtig bei der
\ac{URI}-Path Option (siehe Abschnitt~\ref{option:urioptions} \nameref{option:urioptions}).
In \cite{draft-ietf-core-coap-13} wird festgelegt, welche Optionen mehrfach in einer Nachricht
enthalten sein können und welche nicht.
Kommt eine Option mehrfach vor, welche nicht mehrfach vorkommen darf, muss sie als nichtverstandene
Option behandelt werden.

\subsubsection{Längen und Standardwerte}
Wie in Abbildung \ref{table:coapOptions} zu sehen ist, haben fast alle Optionen
eine Unter-/und Oberbeschränkung ihrer Länge. Hält eine erhaltene Option diesen
Wertebereich nicht ein, so muss sie ebenso als nichtverstandene Option behandelt
werden.
Optionen können Standardwerte besitzen. Möchte man diese Werte verwenden, so
muss die Option nicht extra übermittelt werden.
\subsection{Die Optionen im Einzelnen}
\subsubsection{Uri-Host, Uri-Port, Uri-Path, Uri-Query}
\label{option:urioptions}
% 0011 3 host 
% 0111 7 port 
% 1011 11 path 
% 1111 15 query
Die vier Uri-Optionen dienen dem genauen spezifizieren an welchen Endpoint,
identifiziert mit einer eindeutigen URI, die CoAP-Nachricht gesendet werden
soll. Im Draft sind genaue Regeln spezifiziert, wie aus einer URI die
entsprechenden Optionen erzeugt werden und ebenso der Rückweg. Auf die Regeln
möchten wir hier aber nicht weiter eingehen. Ein Beispiel soll an dieser Stelle genügen. Aus der
URI\\
\verb!coap://[fe80::d2df:9aff:fe72:75bb]:5683/wohnzimmer/temp?location=bottom!\\
würden
\begin{itemize}
  \item eine Uri-Host Option mit dem Inhalt "'fe80::d2df:9aff:fe72:75bb"' ,
  \item eine Uri-Port Option mit dem Inhalt "'5683"',
  \item zwei Uri-Path Optionen, jeweils mit dem Inhalt "'wohnzimmer"' und "'temp"' und
  \item eine Uri-Query Option mit dem Inhalt "'location=bottom"' entstehen.
\end{itemize}   
Alle 4 Optionen sind kritisch und Unsafe-to-Forward. Wie auch schon im Beispiel
zusehen ist, dürfen die Optionen Uri-Path und Uri-Query mehrfach vorkommen.
\subsubsection{ETag}
\label{option:etag}
% 100 4 ETag Repeatable
Der ETag ist eine Bytesequenz und identifiziert eine lokale Ressource auf einem
Server. Dieser Server kann den ETag mit ihm möglichen Mitteln erzeugen,
beispielsweise durch Hashing, Bildung eines Zeitstempels oder Nutzen einer
Versionierung. Allerdings sollte er je nach Anwendungsumgebung nicht
zu groß werden. Der Empfänger eines solchen ETags muss die Bytesequenz nicht
selber auswerten können.

Bei dieser Option ist es wichtig zu unterscheiden, ob sie in einer Request- oder
einer Response-Nachricht auftaucht. Als Option einer Response-Nachricht dient
sie als lokaler Bezeichner für die auf dem Server angefragte Repräsentation
einer Ressource. Dementsprechend darf die Option in einer Response Nachricht nur einmal vorkommen.

Ein Client kann diesen ETag, und möglicherweise zuvor ebenso erhaltene ETags
dieser Ressource, nutzen, um möglicherweise Traffic zu sparen. Hängt der Client an
seine Request-Nachrichten seine ihm bekannten ETags an, so kann der Server
unter Umständen mit einem \verb!2.03! (Valid)-Code signalisieren, dass die
Repräsentation eine der vom Client mitgelieferten Bezeichner der Ressource
entspricht. Um bei mehreren mitgelieferten ETags zu unterscheiden, welcher nun
auf dem Server vorliegt, sendet der Server bei seiner Antwort den aktuellen
ETag wieder mit.

Der Vorteil der Nutzung von ETags wird einem schnell klar, wenn man sich eine
Ressource vorstellt, deren Repräsentation mehrere Bytes umfasst. Anstatt immer
die gesamte Repräsentation zu übertragen, kann es ab der 2. Abfrage
beispielsweise reichen, den ETag zu bestätigen, wenn sich die Ressource nicht
geändert hat.

Anhand des ETags soll beispielhaft erklärt werden, wozu die Unterscheidung der
Safe-to-Forward Optionen in Cache-Key und No-Cache-Key dient.
Die ETag Option hat die Nummer $4_{10}=100_{2}$ und ist deshalb keine kritische Option, da die
Anfrage auch ohne Verstehen der Option ausgeführt werden kann. Versteht ein
Server die ETag Option nicht, da er möglicherweise über kein Verfahren verfügt
um den ETag zu generieren, muss er diese Option auch nicht unterstützen können.
Anstelle den ETag auszuwerten, der ihm scheinbar fälschlich zugesendet wurde,
sendet er einfach eine Response Nachricht zurück und ignoriert diese Option.
Ebenso ist die Option auch nicht Unsafe-to-Forward. Ein Proxy muss diese Option
nicht verstehen können, da sie für die Ausführung seiner Aufgabe keine Rolle
spielt. Die NoCacheKey-Bits \verb!3-5! sehen also wie folgt aus: \verb!001!. Dies bedeutet, dass
der ETag zum Cache-Key gehört, da der NoCacheKey-Block nicht komplett mit
Einsen gesetzt ist. Verwendung findet diese Unterscheidung des Cache-Keys
vorallem bei Proxys, die die Option nicht verstehen können. Sind die
Proxys auch dafür zuständig, Repräsentationen für Ressourcen zu cachen, so
fließt diese ETag-Option trotzdem in den Caching-Algorithmus mit ein, obwohl sie
die Option selbst nicht verstehen.
\paragraph{Beispiel}
Im Cache liegt eine Request-Nachricht mit der ETag Option sowie die dazugehörige
Response-Nachricht.
Im Falle einer Anfrage mit dem gleichen Cache-Key (im einfachsten Fall hier ist es nur der ETag),
kann die Response aus dem Cache zurückgeliefert werden.
Ist allerdings der ETag anders, kann die Response-Nachricht aus dem Cache nicht verwendet werden.
Diese Unterscheidung kann der Proxy anhand des Cache-Keys treffen, ohne die ETag Option zu kennen.
\subsubsection{Max-Age}
\label{option:maxage}
Die Option Max-Age hat die Nummer $14_{10}=1110_{2}$, und ist somit unkritisch,
aber Unsafe-to-Forward. Die Option dient dem Caching-Verhalten von Proxys und
muss deshalb auch zwingend von ihnen verstanden werden. Die, maximal einmalig
vorkommende, Option in einer Response-Nachricht besagt, dass diese Response-Nachricht
maximal dem Optionswert in Sekunden entsprechende Zeit im Cache lag.
Der Wert dieser Option kann zwischen 0 bis zu $2^{32} - 1$~Sekunden (ca.
136,2~Jahren) variieren.
Ist die Option nicht in einer Response enthalten, gilt der Standardwert von 60~Sekunden.
\subsubsection{Location-Path, Location-Query}
\label{option:locationoptions}
Die Location-Path-Option mit der Nummer $8_{10}=1000_{2}$, und die
Location-Query-Option mit der Nummer $20_{10}=10100_{2}$, sind unkritische,
Safe-to-Forward und wiederholbare Optionen. Sie werden benötigt, um relative
Pfade anzugeben. Dies wird beispielsweise notwendig, wenn ein Client ein
POST-Request ausführt und der Server eine neue Ressource anlegt. Dabei gibt er
mit den Location-*-Optionen den relativen Pfad zu der neuen Ressource an.

\subsubsection{Proxy-Uri, Proxy-Scheme}
\label{option:proxyoptions}
Um CoAP die Möglichkeit zu geben, auch in andere Protokolle, wie zum Beispiel
HTTP, Anfragen zu senden, ist es vorgesehen, sogenannte Forward-Proxys
einzusetzen. Diese senden dann die Anfrage weiter oder geben eine zur
Anfrage passende, im Cache vorhandene, Antwort zurück.
In der Proxy-Uri-Option steht ein absoluter Pfad zu der gewünschten Ressource.
Anders als bei den Uri-*-Optionen wird hier die gesamte URI in eine Option
geschrieben. Zudem ist es nicht erlaubt, Uri-*-Optionen und die Proxy-Uri-Option
in einer Nachricht zu verwenden. Erhält ein Endpoint eine Nachricht mit einer
Proxy-*-Option, ist aber nicht in der Lage oder nicht dafür konfiguriert, die
Pakete weiterzuleiten, muss er den Fehlercode \verb!5.05! (Proxying Not Supported) an den
Client zurücksenden. Proxys, welche die Anfrage weiterleiten können, müssen
über ihr Netzwerk so viele Kentnisse besitzen wie nötig sind, um zu entscheiden, ob sie die
Anfrage nochmals an einen Proxy weiterleiten müssen oder direkt den gewünschten Server
ansprechen.\\\\
Da die Proxy-Uri-Option als nicht wiederholbar markiert ist, kann es passieren,
dass die Länge von 1034 nicht ausreicht, um den absoluten Pfad anzugeben. Ebenso
ist es denkbar, dass mehrere Proxys existieren, welche in verschiedene
Protokolle weiterleiten können, um eine Ressource anzusprechen. Ist dies der
Fall, so kann mit Hilfe der Proxy-Scheme-Option gearbeitet werden. Statt nun
den gesamten absoluten Pfad in eine Option schreiben zu müssen, steht nur der
Scheme der absoluten URI in der Proxy-Scheme-Option und der Rest der URI
wird, wie gewohnt, mit den Uri-*-Optionen ausgedrückt.
Da die Uri-*-Optionen aber keine Fragmente unterstützen, können diese bei
solchen Anfragen nicht mit angegeben werden.

Diese zwei Optionen sind als kritisch und Unsafe-to-Forward markiert,
was uns ein Blick auf die Nummern $35_{10}=100011_{2}$ und $39_{10}=100111_{2}$
schnell verrät.

\subsubsection{Content Format}
\label{option:contentformat}
Server können unter Umständen verschiedene Formate zur Darstellung einer
Repräsentationen ihrer Ressource anbieten.
Um dem Client mitzuteilen, welches Format nun im Payload einer Response-Nachricht enthalten ist,
gibt er die Content-Format-Option mit der Nummer $12_{10}=1100_{2}$ an.
Im Optionswert steht einer der in Abbildung \ref{table:contentformat}
aufgeführten IDs.
\begin{figure}[htbp]
   \centering
   \begin{minipage}{.94\textwidth}
   \lstinputlisting[basicstyle=\footnotesize]{sources/tableContentFormats.txt}
   \caption{CoAP Content Formats}
   \end{minipage}
   \label{table:contentformat}
\end{figure}
\subsubsection{Accept}
\label{option:accept}
Nun kann es aber vorkommen, dass ein Client manche Formate aus der Abbildung \ref{table:contentformat} 
nicht verarbeiten kann. Um dem Server mitzuteilen, welche Formate er untersützt, kann er die
Accept-Option nutzen. 
Die Option darf wiederholt werden.
Das gibt dem Client die Möglichkeit mehrere bevorzugte Formate anzugeben. 
Wie einem bei betrachten der Optionsnummer $16_{10}=10000_{2}$ auffällt, ist die Option nicht kritisch.
Es können also drei verschiedene Situationen auftreten:
\begin{description}
\item[Server versteht die Option nicht] Da die Option nicht kritisch ist, führt der Server die Anfrage aus und
ignoriert dabei die Accept-Option des Clients. Dies kann zur Folge haben, dass
der Client ein Format zurückbekommt, dass er nicht verarbeiten kann.
\item[4.06 Not Acceptable] Der Server versteht die Option, kann aber keine der vom Client bevorzugten
Formate anbieten. Er sendet als Antwort eine Fehlermeldung zurück, in der er
angibt, dass er keine der gewünschten Formate zur Verfügung hat.  
\item[2.05 Content] Versteht der Server die Option, so versucht er die Repräsentation der Ressource
in den Formaten anzugeben, die der Client bevorzugt. Dabei ist die Reihenfolge
der Optionen entscheidend. Das zuerst genannte ist ebenso auch das vom Client
am meisten priorisierteste Format.
\end{description}

\subsubsection{Bedingte Anfragen}
\label{option:ifmatch}
% 0001 1 ifmatch Repeatable
In \ac{CoAP} ist es auch möglich, bedingte Anfragen zu stellen. Sind gewünschte
Bedingungen erfüllt, so führt der Server die Anfragen aus, als gäbe es
diese Bedingungen nicht. Sind diese nicht erfüllt, so muss er mit dem
Fehlercode \verb!4.12! (Precondition Failed) antworten.
Zwei Möglichkeiten stehen zur Verfügung:
\begin{description}
\item[If-Match] In dieser Option kann geprüft
werden, ob die angefragte Ressource überhaupt existiert. Dazu wird ein leerer
String als Optionswert mitgesendet oder ein Wert der Ressource verglichen.
Möchte man beispielsweise einem Lost-Update bei einer \verb!PUT!-Anfrage
vorbeugen, so kann in der If-Match-Option ein ETag als Optionswert mitgegeben
werden, was die zuletzt empfangene \verb!GET!-Anfrage geliefert hat. If-Match ist eine
wiederholbare Option. Bei dem Server werden sie als "`Oder"'-Bedingungen
interpretiert, d.h. es muss nur eine der Bedingungen erfüllt werden, um die
Anfrage auszuführen.

\item[If-None-Match] Um paralleles Erzeugen einer Ressource von mehreren Clienten zu
verhindern gibt es die If-None-Match Option. Sie benötigt keinen Wert und ist
auch nicht wiederholbar. Die Bedingung wird wahr, wenn die angefragte Ressource
nicht existiert.
\end{description}
\section{Nachrichtenmodell}

Der Nachrichtenaustausch in \ac{CoAP} basiert auf dem Versand von Nachrichten
via \ac{UDP}.
Jede Nachricht besitzt eine Message-ID, welche zur Duplikaterkennung genutzt
wird.
Zuverlässigkeit kann durch das Versenden einer \acf{CON}-Nachricht erreicht
werden.
Solche Nachrichten werden mit einem Standard-Timeout erneut versendet, bis sie
mit einer \ac{ACK}-Nachricht mit der selben Message-ID vom Empfänger bestätigt
werden oder eine maximale Anzahl von Retransmits erreicht haben.
Wenn der Empfänger die Nachricht nicht verarbeiten kann und diesen Fehler auch
in keiner Antwort ausdrücken kann, dann antwortet der Sender statt mit einem
\ac{ACK} mit einer \acf{RST}-Nachricht.

Wenn eine Nachricht nicht zuverlässig übertragen werden muss, weil sie beispielsweise in einem
regelmäßigen Intervall versendet wird und ein Verlust einer einzelnen Nachricht akzeptiert werden
kann, dann kann sie auch als \acf{NON}-Nachricht verschickt werden.
Diese werden nicht mit einem \ac{ACK} quittiert, besitzen aber eine Message-ID zur
Duplikaterkennung.
Jedoch können sie genau wie \ac{CON}-Nachrichten auch mit einem \ac{RST} beantwortet werden.

\section{Request-Response}

\ac{CoAP} bietet zwei verschiedene Möglichkeiten um auf Requests zu antworten.
Im folgenden werden beide Varianten kurz erläutert und deren Daseinsberechtigung begründet.

\subsection{Piggy-backed Response}
\label{piggybackedResponse}
Bei einer piggy-backed Response wird die eigentliche Antwort direkt mit dem \ac{ACK} des
Requests mitgesendet. Es wird somit unnötiger Netzwerktraffic eingespart.
Im Gegenzug dazu ist es nötig, dass die Antwort quasi sofort verfügbar ist, da für das Senden eines
\ac{ACK}s ein gewisser zeitlicher Rahmen eingehalten werden muss.
Ein Beispiel für diese Art der Kommunikation zeigt
Abbildung~\ref{fig:twoGetRequestsWithPiggyBacking}.

\begin{figure}[htbp]
    \centering
    \begin{minipage}{0.8\textwidth}
    \lstinputlisting[]{sources/twoGetRequestsWithPiggyBacking.txt}
    \caption{Get-Requests mit Piggy-backed Response}
    \end{minipage}
    \label{fig:twoGetRequestsWithPiggyBacking}
\end{figure}


\subsection{Separate Response}
\label{seperateResponse}
\begin{figure}[htbp]
    \centering
    \begin{minipage}{0.75\textwidth}
    \lstinputlisting[]{sources/confirmableRequest_separateResponse.txt}
    \caption{Get-Request mit separater Response}
    \end{minipage}
    \label{fig:confirmableRequest_separateResponse}
\end{figure}

Abbildung~\ref{fig:confirmableRequest_separateResponse} zeigt einen beispielhaften
Nachrichtenaustausch mit separaten Responses.
Die Notwendigkeit der separaten Antworten besteht darin, dass eine Antwort eines Netzteilnehmers
möglicherweise nicht sofort verfügbar ist.
Um das \ac{ACK}-Timeout einzuhalten ist es somit nötig, einen Request vorerst zu bestätigen
und die eigentliche Antwort auf die Anfrage erst zu einem späteren Zeitpunkt abzuschicken.
Weiterhin kann ein Request auch vom Typ \ac{NON} sein, womit ein \ac{ACK} keine
Antwortmöglichkeit darstellt.

Probleme, auf welche man bei dieser Art des Nachrichtenaustauschs stößt, sind das Eintreffen der
Response und des \ac{ACK}s in falscher Reihenfolge, sowie das Request-Response-Matching für
beide Nachrichtenteile der Response.

\subsection{Matching}
\label{Matching}
Das Matching lässt sich mit folgenden Regeln zusammenfassen
\begin{description}
    \item[Allgemein] Response muss vom Endpoint kommen, welcher auch Ziel des Requests war
    \item[Piggy-backed] MessageID und Token müssen übereinstimmen
    \item[Separate] nur Token muss übereinstimmen, ggf. MessageID für eine ACK-Message
\end{description}
Der Token ist elementarer Bestandteil jeder Nachricht, er kann 0 bis 8~Byte lang sein und wird
vom Client generiert.
Er dient der Zuordnung von Responses zu den entsprechenden Requests unabhängig von den
eigentlichen Nachrichten.
Jeder Request führt somit (ggf. implizit) einen Token mit.
Responses auf einen Request können somit eindeutig über die Message-ID, den Token und das Tupel aus
Quelle und Ziel der Nachricht identifiziert werden.

\subsection{Discover}
\label{Discover}
Um vorhandene Ressourcen eines Servers zu erfahren, bietet \ac{CoAP} einen
Discovery Mechanismus.
Als Bedingung für das "`Entdecken"' eines Servers muss dieser lediglich auf dem
Standardport von \ac{CoAP}(5683) lauschen. Je nachdem, welche Protokolle \ac{CoAP}
zugrunde liegen, kann dann mittels Unicast, Multicast oder Broadcast der oder
die Server abefragt werden. Speziell für das Discover wurde der \ac{URI}-Path
\verb!/.well-known/core! reserviert. Dieser kann mittels einer \verb!GET!-Anfrage
abgerufen wurden. Die Antwort enthält dann ein
Linkformat (nach \cite{rfc6690}), in dem die angebotenen Ressourcen beschrieben
sind.

% Reliablility(4.2/4.3) and Congestion Control draft 4.7 in Verbindung mit 4.8 Transmission Parameters
% Freshness Modell
% Validation Model
% URI ( eher kurz gehalten denk ich )
% Discovery! ( IPv6 no Broadcast ) (Groupcomm?)
\section{Erweiterungen}
\ac{CoAP} wie es in \cite{draft-ietf-core-coap-13} beschrieben ist,
bietet die Möglichkeit \ac{REST}ful-Dienste in einem Low-Power-Netzwerk zu integrieren.
Doch neben der Entwicklung des Protokolls selbst findet man auch viele Ideen auf
dem \ac{IETF}-Datatracker \footnote{\url{http://datatracker.ietf.org/wg/core/}}.
Einige Überlegungen betreffen die Integration von \ac{CoAP} in andere Netzwerke, wie
zum Beispiel: "`Transport of \ac{CoAP} over \ac{SMS}, \ac{USSD} and \ac{GPRS}"'
(siehe \cite{draft-becker-core-coap-sms-gprs-03}).
Andere Drafts befassen sich mit Sicherheitsaspekten
(siehe \cite{draft-bormann-core-ipsec-for-coap-00}), oder noch genauer mit der Thematik wie man
mit schlafenden Knoten in einem Sensornetz umgehen soll
(siehe \cite{draft-rahman-core-sleepy-01} und \cite{draft-rahman-core-sleepy-problem-statement-01}) und viele
weitere mehr.
Als Standard-Erweiterungen haben sich die drei folgenden herausgestellt.
\subsection{Block - nach \cite{draft-ietf-core-block-10}}
Für kleine Payloadgrößen muss man sich in \ac{CoAP} keine Gedanken über
Fragmentierung unterliegender Schichten machen. Hat man jedoch größere
Datenmengen zu übertragen, so wird einem durch Abbildung \ref{table:FramePayload}
schnell bewusst, dass auch beim Einsatz von \ac{CoAP} mit eventuell vielen Optionen nicht viel
Platz für den eigentlichen Payload bleibt. Sprengt man den 127-Byte-Rahmen, der vom
\ac{PHY} vorgegeben ist, so müssen die unteren Schichten eine
Fragmentierung durchführen. Um dies zu vermeiden, gibt es die Block-Erweiterung
in \ac{CoAP}. Hier wird die Fragmentierung auf die Applikationsschicht angehoben.
\begin{figure}[htbp]
    \centering
    \begin{minipage}{.61\textwidth}
    \lstinputlisting[basicstyle=\footnotesize]{sources/PacketPayload2.txt}
    \caption{Aufteilung des 802.15.4 Frames}
    \end{minipage}
    \label{table:FramePayload}
\end{figure}
Dies bringt den Vorteil, dass die unteren Schichten, sprich die Adaptions- oder \ac{IP}-schicht,
nicht damit belastet werden. Ein weiterer Vorteil ist das Acknowledgment der
einzelnen Pakete. Tritt ein Fehler auf, so muss nicht das gesamte Paket
wiederholt werden, sondern nur ein Teil davon.\\\\
Um den Blocktransfer zu beschreiben, werden 3 Attribute in der Block-Option
eingeführt.
\begin{description}
 \item[SZX] 
 Die im jeden Request oder Response übermittelte Paketgröße. Sechs verschiedene
 Größen sind vorgesehen, von $2^4=16$ bis $2^{10}=1024$~Bytes. Die Möglichkeit
 von sehr hohen Paketgrößen ist für den Einsatz von \ac{CoAP} mit anderen Schichten
 gedacht, als beispielsweise mit dem 802.15.4/\ac{6LoWPAN}/\ac{UDP} Stack. Bis auf den 
 letzten Block einer Blockübertragung muss der Payload genau diese Größe annehmen.
 Die Paketgröße wird in einem 3-Bit großen Feld angegeben, in der immer nur der 
 Zweierexponent abzüglich 4 eingetragen ist.
 \item[M-Flag] Dieses Attribut gibt an, ob es sich um das letzte Paket handelt, beziehungsweise
 ob noch mehr Pakete vorhanden sind. Die Information wird in einem Bit kodiert. 
\item[NUM] Die Blocknummer gibt an, welches Paket übermittelt, beziehungsweise angefragt,
 wurde. Sie beginnt bei 0 und kann bis zu 1048576 ($2^{20}$) laufen. Die
 Paketnummer wird entweder in 4,12 oder 20~Bit geschrieben.
\end{description}
\begin{figure}[htbp]
    \centering
    \begin{minipage}{.83\textwidth}
    \lstinputlisting[basicstyle=\footnotesize]{sources/BlockOption.txt}
    \caption{Optionen für den Blocktransfer}
    \end{minipage}
    \label{table:blockoption}
\end{figure}
Wie in Abbildung~\ref{table:blockoption} zu sehen ist, werden zwei Block-Optionen benötigt.
Die \verb!Block1!-Option wird verwendet, wenn der Client Daten im Block-Modus zu dem Server
senden möchte oder muss, weil der Server es verlangt.
Die \verb!Block2!-Option wird verwendet, wenn bei einer \verb!GET!-Anfrage an den Server die
Antwort im Block-Modus zurückgeschickt wird.
Die genaue Belegung der Attribute je nach Anfrage soll hier allerdings nicht
genauer erläutert werden.

Die Block-Option selbst hat keine Beschreibung, wieviele Pakete noch kommen. Es
gibt nur eine Paketnummer, die Blockgröße und ein Flag, was besagt, ob noch mehr
Blöcke vorhanden sind. Da es in manchen Anwendungen notwendig ist, wurde ebenso
auch die optionale Size-Option eingeführt. Mit Hilfe dieser Option kann man die
Größe des eigentlichen Payloads angeben.

Die Einführung der Block-Option hebt keinesfalls das \ac{REST}-Prinzip auf, denn
weiterhin beschreiben alle Anfragen sich soweit, dass sie vom Server ausgeführt
werden können, ohne sich einen Status zu merken.

\subsection{Observe - nach \cite{draft-ietf-core-observe-07}}
In Anbetracht der Tatsache, dass \ac{CoAP} dafür entwickelt ist, um mit möglichst wenig Overhead
einen \ac{REST}ful Dienst zu ermöglichen, treten Probleme bei der Anwendung von \ac{CoAP} in
Sensornetzen auf.
Dort möchte man beispielsweise recht schnell mitbekommen, wann sich eine Tür geöffnet hat.
Es bleibt also nur die Möglichkeit in engen Abständen den Zustand der Tür abzufragen.
Gehen wir ein Stück weiter und stellen uns die Tür zu einem sehr selten frequentierten Raum vor, so
wird sehr viel Aufwand und Netzlast erzeugt, um in einem aktzeptablen Abstand informiert zu werden,
dass sich der Status der Tür geändert hat.
Problem hier ist der Mechanismus der \ac{REST}-Struktur, in der immer der Client eine Anfrage
abschickt und eine Antwort erhält und die Kommunikation damit abgeschlossen ist.
\subsubsection{Überwachen von Ressourcen}
Mit der Idee der Überwachung wird ein neuer Mechanismus in \ac{CoAP} eingeführt.
Anstatt nur einmalig eine Antwort vom einen Server auf eine Anfrage zu bekommen, ist es möglich sich
von dem Server über die Repräsentation einer Ressource informieren zu lassen.
Das \ac{ODP}\cite{GAMMAETAL} wurde in \ac{CoAP} integriert mit der Observe-Option, welche in
Abbildung~\ref{table:observeoption} dargestellt ist.
 \begin{figure}[htbp]
    \centering
    \begin{minipage}{.94\textwidth}
    \lstinputlisting[basicstyle=\footnotesize]{sources/ObserveOption.txt}
    \caption{Die Observe-Option}
    \end{minipage}
    \label{table:observeoption}
\end{figure}
Alle \ac{REST}-Eigenschaften bleiben dabei erhalten, doch ist es nun einem Client möglich auf eine
Anfrage mehrere Antworten zu bekommen.
Um dies zu realisieren, gibt es die Möglichkeit sich bei dem Server als Observer in eine Liste
eintragen zu lassen.
Ist man in der Liste, so erhält man von nun an bei Statuswechseln, gewissen abgelaufenen
Zeitabständen eine Notification über die neue/aktuelle Repräsentation der überwachten Ressource.
Die genaue Implementation soll in diesem Dokument hier nicht dargestellt werden, doch sei darauf
hingewiesen, das es ebenso möglich ist die Observe-Option mit der ETag-Option zu kombinieren.
\subsubsection{Probleme mit Safe-to-Forward}
Seit dem 6. Februar 2013 ist eine Diskussion in der Mailingliste zu beobachten, bei der es darum
geht, dass es Probleme seit dem Draft-12 (siehe \cite{draft-ietf-core-coap-12}) von \ac{CoAP} gibt.
In diesem Draft wurde die Unterscheidung in Un-/Safe-to-Forward eingeführt.
Diese Unterscheidung führt nun dazu, dass Proxys, welche zwar die Observe Funktion verstehen, aber
möglicherweise andere Optionen nicht, welche die Überwachung aber beeinflussen.
Ebenso tritt es dadurch auf das Clienten sich gegenseitig aus der Liste der Observer löschen.
Derzeit tendiert die Diskussion wohl dahin ein weitere Unterscheidung der Optionen einzubauen.
Neben dem Cache-Key, gibt es womöglich demnächst ein Observe-Key, welcher genutzt werden soll um
die oben angesprochenen Probleme zu beheben.
\subsection{Groupcomm nach \cite{draft-ietf-core-groupcomm-05}}
\subsubsection{Notwendigkeit}
Für \ac{CoAP} selbst wird nicht geklärt, wie eine Gruppenkommunikation auszusehen hat.
Eine Gruppenkommunikation ist jedoch in der Praxis unerlässlich.
Ein Beispiel hierfür wäre die Einteilung von Sensoren/Aktoren in einem Netz nach Typ oder Standort.
Eine mögliche Gruppe von Aktoren wären alle Lichtschalter im Haus oder alle vorhandenen
Thermostate.
Es wird Wert darauf gelegt, dass sowohl in \ac{CoAP} als auch in \ac{IP}-Multicasts keine
Änderungen hierfür vorgenommen werden müssen.
Insbesondere wie einzelne Protokolle für verschiedene Szenarien miteinander eingesetzt werden
müssen wird in dem Dokument erläutert.
\subsubsection{Richtlinien}
Da \ac{IP}-Multicasts weit verbreitet sind, bildet dieser Mechanismus die Grundlage für die
Gruppenkommunikation.
Um Multicasts über mehrere Subnetze zu ermöglichen ist es nötig Routing-Protokolle oder
Forwarding-Protokolle einzusetzen.
Für den Einsatz denkbar wäre hier beispielsweise \ac{RPL} oder \ac{MPL}.
Weiterhin wird noch ein sogenanntes Listener-Protokoll benötigt, um einzelne Geräte Gruppen
zuzuordnen.\\\\
Die Gruppen-\ac{URI} muss in der Nachricht die Request-\ac{URI} sein.
Der Authority-Teil muss eine Gruppen-\ac{IP}-Multicast-Adresse oder ein Hostname sein, welcher zu einer
solchen aufgelöst werden kann.
Folgende Beispiele für mögliche Gruppen-\ac{URI}s werden aufgelistet:
\begin{description}
\item[all.bldg6.example.com] all nodes in building 6
\item[all.west.bldg6.example.com] all nodes in west wing, building 6
\item[all.floor1.west.bldg6.examp...] all nodes in floor 1, west wing, building 6
\item[all.bu036.floor1.west.bldg6...] all nodes in office bu036, floor1, west wing, building 6
\end{description}
Der Gruppen-\ac{FQDN} muss via \ac{DNS} aufgelöst werden können.\\\\
Weiterhin wird geklärt, wie ein Knoten die Mitgliedschaft einer Multicast-Gruppe ankündigen kann.
Es wird die Vorgehensweise für die Protokolle \ac{MLD}, \ac{RPL}, \ac{MPL} kurz erläutert, was im
Umfang dieses Dokuments hier aber zu weit führen würde.\\\\
Für Multi-LoWPAN-Szenarien wird empfohlen, Filtering-Regeln einzuführen, um unnötige Netzlast zu
vermeiden. 
\paragraph{Filtern durch \ac{6LBR} mit Hilfe von Routinginformationen}
Dies ermöglicht es dem \ac{6LBR}, Multicast-Nachrichten nur in Netze weiterzuleiten, in denen auch
ein Empfänger für die Nachricht existiert
\paragraph{Filtern durch \ac{6LBR} mit Hilfe von \ac{MLD}-Reports}
Gleich dem vorherigen Verfahren, jedoch basierend auf \ac{MLD}-Reports.
\paragraph{Filtern durch \ac{6LBR} mit Hilfe von Einstellungen}
Nutzen von Filter-Tabellen mit Blacklists bzw. Whitelists für alle \ac{6LBR} bzw. für spezifische
\ac{6LBR}.
\subsubsection{Probleme}
Nach draft-13 (siehe \cite{draft-ietf-core-coap-13}) von \ac{CoAP} muss die Gruppenkommunikation im
\ac{NoSec}-Modus erfolgen.
Weiterhin darf kein \ac{DTLS}-gesichertes \ac{CoAP} und auch kein \ac{IPSec} verwendet werden.
Aus diesen Gründen müssen Sicherheitsvorkehrungen auf anderen Schichten vorgenommen werden.
Für die Zukunft wird darauf spekuliert, dass \ac{DTLS}-basierte \ac{IP}-Multicasts oder andere
Herangehensweisen die Sicherheitsprobleme lösen könnten.
Aktuell ist aber noch kein wirklicher Lösungsansatz für diese Probleme vorhanden. % Statt/zusätzlich Groupcomm, [draft-li-core-coap-payload-length-option] ?
\section{Zusammenfassung}
\ac{CoAP} implementiert eine \ac{REST}ful Architektur mit möglichst wenig Overhead,
Was es vorallem im Einsatz mit Sensornetzen interessant macht: Zum einen ist es
möglich, Sensorwerte abzufragen, zum anderen bietet es auch die Möglichkeit, Werte
an den Endpoints zu ändern. Durch das Verwenden von \ac{UDP} als Transportschicht,
ist zwar die Zuverlässigkeit nicht mehr gesichert, doch spart man einiges an
Overhead gegenüber \ac{TCP}. Ist man daran interessiert, dennoch
Zuverlässigkeitsmechanismen zu benutzen, so bietet auch dort \ac{CoAP}
Möglichkeiten. Weiterhin bietet das Verwenden einer \ac{REST}-Architektur leichte
Anbindung an andere \ac{REST}-Netzwerke wie z.B. \ac{HTTP}. Ein Mapping von \ac{CoAP}
auf \ac{HTTP} wurde in diesem Dokument nicht beschrieben, doch ist dies auch bereits
in \cite{draft-ietf-core-coap-13} beschrieben. Durch Proxys und den Einsatz von den
oben beschriebenen Proxy-*-Optionen, ist es möglich auch aus dem \ac{CoAP} Netzwerk
heraus oder hinein Anfragen zu stellen.

Vorallem wenn man die Mailingliste von \ac{CoAP} verfolgt, stellt man fest, dass
es zwar doch noch an einigen Stellen zu nicht-vorhergesehenen Fehlern kommt, doch
durch die aktive Arbeit mehrerer Leute wird dies ausführlich diskutiert und
Änderungen werden im Draft festgehalten. Trotz dieser auftretenen Probleme ist
\ac{CoAP} in einem Zustand, in dem es implementiert und bearbeitet werden kann.
Bei der Implementierung allerdings scheitert es derzeit. Nur für wenige Sprachen
sind derzeit Implementierungen einzelner Leute im Internet zu finden. Immerhin
für die Programmiersprache C gibt es eine aktive
Gruppe\footnote{\url{http://libcoap.sourceforge.net/}}, welche versucht ihre
Implementierung auf dem aktuellen Stand zu halten.

Um \ac{CoAP} hinsichtlich Effizienz zu beurteilen fehlt es dadurch leider an
Testberichten. Doch die Aufmerksamkeit steigt und nachdem nun die Entwicklung
einen Stand erreicht hat, indem es sich lohnt, eine Bibliothek zu programmieren,
wird es schon möglich sein, bessere Aussagen zu treffen. Aus theorethischer Sicht
macht CoAP definitiv Sinn und hat gute Aussichten für die Zukunft, vorallem im
Bereich Sensornetzen und anderen Low-Power \ac{M2M}-Netzwerken.

 
\begin{appendix}
\section{Tabellen}
\begin{figure}[htbp]
    \centering
    \begin{minipage}{0.4\textwidth}
    \lstinputlisting[basicstyle=\scriptsize]{sources/tableMethodCodes.txt}
    \caption{Request Codes}
    \end{minipage}
    \label{table:requestCodes}
\end{figure}

\begin{figure}[htbp]
    \centering
    \begin{minipage}{0.75\textwidth}
    \lstinputlisting[basicstyle=\footnotesize]{sources/tableResponseCodes.txt}
    \caption{Response Codes}
    \end{minipage}
    \label{table:responseCodes}
\end{figure}


\end{appendix}
% Glossar ?
\renewcommand{\bibname}{Quellenverzeichnis} %aendert den Namen des Literaturverzeichnisses
\bibliography{rfc,drafts,books}
\bibliographystyle{styles/alphadin}
\addcontentsline{toc}{section}{Quellenverzeichnis}

\dedication{ }
%\maketitle\thispagestyle{empty}
\end{document}
\usepackage[plainpages=false]{hyperref}
