\section{Optionen}
Im Abschnitt~\ref{optionFormat} \nameref{optionFormat} ist bereits der Aufbau einer Option erläutert.
In diesem Kapitel soll einmal genau der Aufbau der Optionsnummer betrachtet werden und ebenso
welche verschiedenen Optionen CoAP zur Verfügung stellt und wozu diese notwendig sind.
\begin{figure}[htbp]
    \centering
    \begin{minipage}{.97\textwidth}
    \lstinputlisting[basicstyle=\footnotesize]{sources/tableOptions.txt}
    \caption{CoAP Optionen}
    \end{minipage}
    \label{table:coapOptions}
\end{figure}
\subsection{Eigenschaften}
\begin{figure}[htbp]
    \centering
    \begin{minipage}{.5\textwidth}
    \lstinputlisting[basicstyle=\footnotesize]{sources/optionNumberMask.txt}
    \caption{Optionsnummernmaske}
    \end{minipage}
    \label{table:optionnumbermask}
\end{figure}
\subsubsection{Critical / Elective}
Ist eine Option als kritisch markiert, muss diese verstanden werden, andernfalls führt sie zu
folgenden drei Fehlerfällen:
\begin{description}
  \item [\ac{CON}-Request] 
  Kann die Option bei einer \ac{CON}-Request-Nachricht nicht verstanden
werden, muss sie mit dem Fehlercode \verb!4.02! (Bad Option) beantwortet werden.
Zusätzlich sollte im Payload einer solchen Antwort eine kurze,
menschlich lesbare, \ac{UTF-8} kodierte Nachricht stehen, die den Fehler beschreibt.
Beispielsweise bei welcher Option der Fehler überhaupt auftrat und was nicht
verstanden wurde oder man beispielsweise die Option überhaupt nicht kennt.
  \item [\ac{CON}-Response]
  Kann bei einer \ac{CON}-Response oder einer durch
Piggy-backed-Response (siehe Abschnitt~\ref{piggybackedResponse} \nameref{piggybackedResponse}) 
erhaltenen Nachricht eine kritische Option nicht verstanden werden, muss
diese mit einer \ac{RST}-Nachricht beantwortet werden und die Response selbst aber ignoriert
werden.
  \item [\ac{NON}-Nachricht]
  Nachrichten mit einer oder mehreren nichtverstandenen kritischen Optionen müssen ignoriert
  werden.
  Der Empfänger einer solchen fehlerhaften oder nicht verstandenen Option kann diese Nachricht auch
  durch eine \ac{RST}-Nachricht beantworten, muss sie aber dennoch ignorieren. 
\end{description}
Kann die Option nicht verstanden werden und ist aber nicht kritisch, so wird die Option ignoriert,
die Nachricht allerdings nicht.

Ob eine Option kritisch ist oder nicht, entscheidet sich durch das niedrigstwertige Bit der
Optionsnummer, wie in Abbildung~\ref{table:optionnumbermask} zu sehen. Ist es gesetzt, so ist die Option kritisch, sonst nicht. Damit sind alle Optionen mit
einer ungeraden Optionsnummer kritisch.

\subsubsection{Proxy Unsafe/Safe}
Proxys müssen nicht alle Optionen verstehen können. Aus diesem Grund, ist es für diese Gruppe von
Endpoints nicht relevant, ob die Optionen kritisch sind oder nicht sind. An dieser Stelle wird
zwischen Safe-to-Forward oder Unsafe-to-Forward unterschieden. Versteht ein Proxy also eine Option
nicht, die mit Unsafe-to-Forward markiert ist, muss er diese mit einem \verb!4.02! (Bad Option)
Fehlercode beantworten.
Safe-to-Forward Optionen werden von Proxys nochmals unterschieden. Nämlich ob sie zum Cache-Key gehören oder
nicht.
Wozu diese zusätzliche Unterscheidung dient, wird bei der Erklärung der ETag-Option (siehe Abschnitt~\ref{option:etag} \nameref{option:etag})
verdeutlicht.
Das zweit-niedrigstwertige Bit der Optionsnummer entscheidet darüber, ob die Option Safe-to-Forward
ist oder nicht, vergleiche Abbildung~\ref{table:optionnumbermask}. Ist es gesetzt, so ist die Option Unsafe-to-Forward. Ist es nicht gesetzt, so
unterscheiden die 3 nächsthöheren Bits darüber, ob die Option zum Cache-Key gehört oder nicht. Ist
eines der 3 Bits nicht gesetzt, so gehört die Safe-to-Forward Option zum Cache-Key.

\subsubsection{Repeatable}
Manche Optionen können in \ac{CoAP} mehrfach vorkommen. Dies ist zum Beispiel wichtig bei der
\ac{URI}-Path Option (siehe Abschnitt~\ref{option:urioptions} \nameref{option:urioptions}).
In \cite{draft-ietf-core-coap-13} wird festgelegt, welche Optionen mehrfach in einer Nachricht
enthalten sein können und welche nicht.
Kommt eine Option mehrfach vor, welche nicht mehrfach vorkommen darf, muss sie als nichtverstandene
Option behandelt werden.

\subsubsection{Längen und Standardwerte}
Wie in Abbildung \ref{table:coapOptions} zu sehen ist, haben fast alle Optionen
eine Unter-/und Oberbeschränkung ihrer Länge. Hält eine erhaltene Option diesen
Wertebereich nicht ein, so muss sie ebenso als nichtverstandene Option behandelt
werden.
Optionen können Standardwerte besitzen. Möchte man diese Werte verwenden, so
muss die Option nicht extra übermittelt werden.
\subsection{Die Optionen im Einzelnen}
\subsubsection{Uri-Host, Uri-Port, Uri-Path, Uri-Query}
\label{option:urioptions}
% 0011 3 host 
% 0111 7 port 
% 1011 11 path 
% 1111 15 query
Die vier Uri-Optionen dienen dem genauen spezifizieren an welchen Endpoint,
identifiziert mit einer eindeutigen URI, die CoAP-Nachricht gesendet werden
soll. Im Draft sind genaue Regeln spezifiziert, wie aus einer URI die
entsprechenden Optionen erzeugt werden und ebenso der Rückweg. Auf die Regeln
möchten wir hier aber nicht weiter eingehen. Ein Beispiel soll an dieser Stelle genügen. Aus der
URI\\
\verb!coap://[fe80::d2df:9aff:fe72:75bb]:5683/wohnzimmer/temp?location=bottom!\\
würden
\begin{itemize}
  \item eine Uri-Host Option mit dem Inhalt "'fe80::d2df:9aff:fe72:75bb"' ,
  \item eine Uri-Port Option mit dem Inhalt "'5683"',
  \item zwei Uri-Path Optionen, jeweils mit dem Inhalt "'wohnzimmer"' und "'temp"' und
  \item eine Uri-Query Option mit dem Inhalt "'location=bottom"' entstehen.
\end{itemize}   
Alle 4 Optionen sind kritisch und Unsafe-to-Forward. Wie auch schon im Beispiel
zusehen ist, dürfen die Optionen Uri-Path und Uri-Query mehrfach vorkommen.
\subsubsection{ETag}
\label{option:etag}
% 100 4 ETag Repeatable
Der ETag ist eine Bytesequenz und identifiziert eine lokale Ressource auf einem
Server. Dieser Server kann den ETag mit ihm möglichen Mitteln erzeugen,
beispielsweise durch Hashing, Bildung eines Zeitstempels oder Nutzen einer
Versionierung. Allerdings sollte er je nach Anwendungsumgebung nicht
zu groß werden. Der Empfänger eines solchen ETags muss die Bytesequenz nicht
selber auswerten können.

Bei dieser Option ist es wichtig zu unterscheiden, ob sie in einer Request- oder
einer Response-Nachricht auftaucht. Als Option einer Response-Nachricht dient
sie als lokaler Bezeichner für die auf dem Server angefragte Repräsentation
einer Ressource. Dementsprechend darf die Option in einer Response Nachricht nur einmal vorkommen.

Ein Client kann diesen ETag, und möglicherweise zuvor ebenso erhaltene ETags
dieser Ressource, nutzen, um möglicherweise Traffic zu sparen. Hängt der Client an
seine Request-Nachrichten seine ihm bekannten ETags an, so kann der Server
unter Umständen mit einem \verb!2.03! (Valid)-Code signalisieren, dass die
Repräsentation eine der vom Client mitgelieferten Bezeichner der Ressource
entspricht. Um bei mehreren mitgelieferten ETags zu unterscheiden, welcher nun
auf dem Server vorliegt, sendet der Server bei seiner Antwort den aktuellen
ETag wieder mit.

Der Vorteil der Nutzung von ETags wird einem schnell klar, wenn man sich eine
Ressource vorstellt, deren Repräsentation mehrere Bytes umfasst. Anstatt immer
die gesamte Repräsentation zu übertragen, kann es ab der 2. Abfrage
beispielsweise reichen, den ETag zu bestätigen, wenn sich die Ressource nicht
geändert hat.

Anhand des ETags soll beispielhaft erklärt werden, wozu die Unterscheidung der
Safe-to-Forward Optionen in Cache-Key und No-Cache-Key dient.
Die ETag Option hat die Nummer $4_{10}=100_{2}$ und ist deshalb keine kritische Option, da die
Anfrage auch ohne Verstehen der Option ausgeführt werden kann. Versteht ein
Server die ETag Option nicht, da er möglicherweise über kein Verfahren verfügt
um den ETag zu generieren, muss er diese Option auch nicht unterstützen können.
Anstelle den ETag auszuwerten, der ihm scheinbar fälschlich zugesendet wurde,
sendet er einfach eine Response Nachricht zurück und ignoriert diese Option.
Ebenso ist die Option auch nicht Unsafe-to-Forward. Ein Proxy muss diese Option
nicht verstehen können, da sie für die Ausführung seiner Aufgabe keine Rolle
spielt. Die NoCacheKey-Bits \verb!3-5! sehen also wie folgt aus: \verb!001!. Dies bedeutet, dass
der ETag zum Cache-Key gehört, da der NoCacheKey-Block nicht komplett mit
Einsen gesetzt ist. Verwendung findet diese Unterscheidung des Cache-Keys
vorallem bei Proxys, die die Option nicht verstehen können. Sind die
Proxys auch dafür zuständig, Repräsentationen für Ressourcen zu cachen, so
fließt diese ETag-Option trotzdem in den Caching-Algorithmus mit ein, obwohl sie
die Option selbst nicht verstehen.
\paragraph{Beispiel}
Im Cache liegt eine Request-Nachricht mit der ETag Option sowie die dazugehörige
Response-Nachricht.
Im Falle einer Anfrage mit dem gleichen Cache-Key (im einfachsten Fall hier ist es nur der ETag),
kann die Response aus dem Cache zurückgeliefert werden.
Ist allerdings der ETag anders, kann die Response-Nachricht aus dem Cache nicht verwendet werden.
Diese Unterscheidung kann der Proxy anhand des Cache-Keys treffen, ohne die ETag Option zu kennen.
\subsubsection{Max-Age}
\label{option:maxage}
Die Option Max-Age hat die Nummer $14_{10}=1110_{2}$, und ist somit unkritisch,
aber Unsafe-to-Forward. Die Option dient dem Caching-Verhalten von Proxys und
muss deshalb auch zwingend von ihnen verstanden werden. Die, maximal einmalig
vorkommende, Option in einer Response-Nachricht besagt, dass diese Response-Nachricht
maximal dem Optionswert in Sekunden entsprechende Zeit im Cache lag.
Der Wert dieser Option kann zwischen 0 bis zu $2^{32} - 1$~Sekunden (ca.
136,2~Jahren) variieren.
Ist die Option nicht in einer Response enthalten, gilt der Standardwert von 60~Sekunden.
\subsubsection{Location-Path, Location-Query}
\label{option:locationoptions}
Die Location-Path-Option mit der Nummer $8_{10}=1000_{2}$, und die
Location-Query-Option mit der Nummer $20_{10}=10100_{2}$, sind unkritische,
Safe-to-Forward und wiederholbare Optionen. Sie werden benötigt, um relative
Pfade anzugeben. Dies wird beispielsweise notwendig, wenn ein Client ein
POST-Request ausführt und der Server eine neue Ressource anlegt. Dabei gibt er
mit den Location-*-Optionen den relativen Pfad zu der neuen Ressource an.

\subsubsection{Proxy-Uri, Proxy-Scheme}
\label{option:proxyoptions}
Um CoAP die Möglichkeit zu geben, auch in andere Protokolle, wie zum Beispiel
HTTP, Anfragen zu senden, ist es vorgesehen, sogenannte Forward-Proxys
einzusetzen. Diese senden dann die Anfrage weiter oder geben eine zur
Anfrage passende, im Cache vorhandene, Antwort zurück.
In der Proxy-Uri-Option steht ein absoluter Pfad zu der gewünschten Ressource.
Anders als bei den Uri-*-Optionen wird hier die gesamte URI in eine Option
geschrieben. Zudem ist es nicht erlaubt, Uri-*-Optionen und die Proxy-Uri-Option
in einer Nachricht zu verwenden. Erhält ein Endpoint eine Nachricht mit einer
Proxy-*-Option, ist aber nicht in der Lage oder nicht dafür konfiguriert, die
Pakete weiterzuleiten, muss er den Fehlercode \verb!5.05! (Proxying Not Supported) an den
Client zurücksenden. Proxys, welche die Anfrage weiterleiten können, müssen
über ihr Netzwerk so viele Kentnisse besitzen wie nötig sind, um zu entscheiden, ob sie die
Anfrage nochmals an einen Proxy weiterleiten müssen oder direkt den gewünschten Server
ansprechen.\\\\
Da die Proxy-Uri-Option als nicht wiederholbar markiert ist, kann es passieren,
dass die Länge von 1034 nicht ausreicht, um den absoluten Pfad anzugeben. Ebenso
ist es denkbar, dass mehrere Proxys existieren, welche in verschiedene
Protokolle weiterleiten können, um eine Ressource anzusprechen. Ist dies der
Fall, so kann mit Hilfe der Proxy-Scheme-Option gearbeitet werden. Statt nun
den gesamten absoluten Pfad in eine Option schreiben zu müssen, steht nur der
Scheme der absoluten URI in der Proxy-Scheme-Option und der Rest der URI
wird, wie gewohnt, mit den Uri-*-Optionen ausgedrückt.
Da die Uri-*-Optionen aber keine Fragmente unterstützen, können diese bei
solchen Anfragen nicht mit angegeben werden.

Diese zwei Optionen sind als kritisch und Unsafe-to-Forward markiert,
was uns ein Blick auf die Nummern $35_{10}=100011_{2}$ und $39_{10}=100111_{2}$
schnell verrät.

\subsubsection{Content Format}
\label{option:contentformat}
Server können unter Umständen verschiedene Formate zur Darstellung einer
Repräsentationen ihrer Ressource anbieten.
Um dem Client mitzuteilen, welches Format nun im Payload einer Response-Nachricht enthalten ist,
gibt er die Content-Format-Option mit der Nummer $12_{10}=1100_{2}$ an.
Im Optionswert steht einer der in Abbildung \ref{table:contentformat}
aufgeführten IDs.
\begin{figure}[htbp]
   \centering
   \begin{minipage}{.94\textwidth}
   \lstinputlisting[basicstyle=\footnotesize]{sources/tableContentFormats.txt}
   \caption{CoAP Content Formats}
   \end{minipage}
   \label{table:contentformat}
\end{figure}
\subsubsection{Accept}
\label{option:accept}
Nun kann es aber vorkommen, dass ein Client manche Formate aus der Abbildung \ref{table:contentformat} 
nicht verarbeiten kann. Um dem Server mitzuteilen, welche Formate er untersützt, kann er die
Accept-Option nutzen. 
Die Option darf wiederholt werden.
Das gibt dem Client die Möglichkeit mehrere bevorzugte Formate anzugeben. 
Wie einem bei betrachten der Optionsnummer $16_{10}=10000_{2}$ auffällt, ist die Option nicht kritisch.
Es können also drei verschiedene Situationen auftreten:
\begin{description}
\item[Server versteht die Option nicht] Da die Option nicht kritisch ist, führt der Server die Anfrage aus und
ignoriert dabei die Accept-Option des Clients. Dies kann zur Folge haben, dass
der Client ein Format zurückbekommt, dass er nicht verarbeiten kann.
\item[4.06 Not Acceptable] Der Server versteht die Option, kann aber keine der vom Client bevorzugten
Formate anbieten. Er sendet als Antwort eine Fehlermeldung zurück, in der er
angibt, dass er keine der gewünschten Formate zur Verfügung hat.  
\item[2.05 Content] Versteht der Server die Option, so versucht er die Repräsentation der Ressource
in den Formaten anzugeben, die der Client bevorzugt. Dabei ist die Reihenfolge
der Optionen entscheidend. Das zuerst genannte ist ebenso auch das vom Client
am meisten priorisierteste Format.
\end{description}

\subsubsection{Bedingte Anfragen}
\label{option:ifmatch}
% 0001 1 ifmatch Repeatable
In \ac{CoAP} ist es auch möglich, bedingte Anfragen zu stellen. Sind gewünschte
Bedingungen erfüllt, so führt der Server die Anfragen aus, als gäbe es
diese Bedingungen nicht. Sind diese nicht erfüllt, so muss er mit dem
Fehlercode \verb!4.12! (Precondition Failed) antworten.
Zwei Möglichkeiten stehen zur Verfügung:
\begin{description}
\item[If-Match] In dieser Option kann geprüft
werden, ob die angefragte Ressource überhaupt existiert. Dazu wird ein leerer
String als Optionswert mitgesendet oder ein Wert der Ressource verglichen.
Möchte man beispielsweise einem Lost-Update bei einer \verb!PUT!-Anfrage
vorbeugen, so kann in der If-Match-Option ein ETag als Optionswert mitgegeben
werden, was die zuletzt empfangene \verb!GET!-Anfrage geliefert hat. If-Match ist eine
wiederholbare Option. Bei dem Server werden sie als "`Oder"'-Bedingungen
interpretiert, d.h. es muss nur eine der Bedingungen erfüllt werden, um die
Anfrage auszuführen.

\item[If-None-Match] Um paralleles Erzeugen einer Ressource von mehreren Clienten zu
verhindern gibt es die If-None-Match Option. Sie benötigt keinen Wert und ist
auch nicht wiederholbar. Die Bedingung wird wahr, wenn die angefragte Ressource
nicht existiert.
\end{description}