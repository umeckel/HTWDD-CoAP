\section{Nachrichtenmodell}

Der Nachrichtenaustausch in \ac{CoAP} basiert auf dem Versand von Nachrichten
via \ac{UDP}.
Jede Nachricht besitzt eine Message-ID, welche zur Duplikaterkennung genutzt
wird.
Zuverlässigkeit kann durch das Versenden einer \acf{CON}-Nachricht erreicht
werden.
Solche Nachrichten werden mit einem Standard-Timeout erneut versendet, bis sie
mit einer \ac{ACK}-Nachricht mit der selben Message-ID vom Empfänger bestätigt
werden oder eine maximale Anzahl von Retransmits erreicht haben.
Wenn der Empfänger die Nachricht nicht verarbeiten kann und diesen Fehler auch
in keiner Antwort ausdrücken kann, dann antwortet der Sender statt mit einem
\ac{ACK} mit einer \acf{RST}-Nachricht.

Wenn eine Nachricht nicht zuverlässig übertragen werden muss, weil sie beispielsweise in einem
regelmäßigen Intervall versendet wird und ein Verlust einer einzelnen Nachricht akzeptiert werden
kann, dann kann sie auch als \acf{NON}-Nachricht verschickt werden.
Diese werden nicht mit einem \ac{ACK} quittiert, besitzen aber eine Message-ID zur
Duplikaterkennung.
Jedoch können sie genau wie \ac{CON}-Nachrichten auch mit einem \ac{RST} beantwortet werden.

\section{Request-Response}

\ac{CoAP} bietet zwei verschiedene Möglichkeiten um auf Requests zu antworten.
Im folgenden werden beide Varianten kurz erläutert und deren Daseinsberechtigung begründet.

\subsection{Piggy-backed Response}
\label{piggybackedResponse}
Bei einer piggy-backed Response wird die eigentliche Antwort direkt mit dem \ac{ACK} des
Requests mitgesendet. Es wird somit unnötiger Netzwerktraffic eingespart.
Im Gegenzug dazu ist es nötig, dass die Antwort quasi sofort verfügbar ist, da für das Senden eines
\ac{ACK}s ein gewisser zeitlicher Rahmen eingehalten werden muss.
Ein Beispiel für diese Art der Kommunikation zeigt
Abbildung~\ref{fig:twoGetRequestsWithPiggyBacking}.

\begin{figure}[htbp]
    \centering
    \begin{minipage}{0.8\textwidth}
    \lstinputlisting[]{sources/twoGetRequestsWithPiggyBacking.txt}
    \caption{Get-Requests mit Piggy-backed Response}
    \end{minipage}
    \label{fig:twoGetRequestsWithPiggyBacking}
\end{figure}


\subsection{Separate Response}
\label{seperateResponse}
\begin{figure}[htbp]
    \centering
    \begin{minipage}{0.75\textwidth}
    \lstinputlisting[]{sources/confirmableRequest_separateResponse.txt}
    \caption{Get-Request mit separater Response}
    \end{minipage}
    \label{fig:confirmableRequest_separateResponse}
\end{figure}

Abbildung~\ref{fig:confirmableRequest_separateResponse} zeigt einen beispielhaften
Nachrichtenaustausch mit separaten Responses.
Die Notwendigkeit der separaten Antworten besteht darin, dass eine Antwort eines Netzteilnehmers
möglicherweise nicht sofort verfügbar ist.
Um das \ac{ACK}-Timeout einzuhalten ist es somit nötig, einen Request vorerst zu bestätigen
und die eigentliche Antwort auf die Anfrage erst zu einem späteren Zeitpunkt abzuschicken.
Weiterhin kann ein Request auch vom Typ \ac{NON} sein, womit ein \ac{ACK} keine
Antwortmöglichkeit darstellt.

Probleme, auf welche man bei dieser Art des Nachrichtenaustauschs stößt, sind das Eintreffen der
Response und des \ac{ACK}s in falscher Reihenfolge, sowie das Request-Response-Matching für
beide Nachrichtenteile der Response.

\subsection{Matching}
\label{Matching}
Das Matching lässt sich mit folgenden Regeln zusammenfassen
\begin{description}
    \item[Allgemein] Response muss vom Endpoint kommen, welcher auch Ziel des Requests war
    \item[Piggy-backed] MessageID und Token müssen übereinstimmen
    \item[Separate] nur Token muss übereinstimmen, ggf. MessageID für eine ACK-Message
\end{description}
Der Token ist elementarer Bestandteil jeder Nachricht, er kann 0 bis 8~Byte lang sein und wird
vom Client generiert.
Er dient der Zuordnung von Responses zu den entsprechenden Requests unabhängig von den
eigentlichen Nachrichten.
Jeder Request führt somit (ggf. implizit) einen Token mit.
Responses auf einen Request können somit eindeutig über die Message-ID, den Token und das Tupel aus
Quelle und Ziel der Nachricht identifiziert werden.

\subsection{Discover}
\label{Discover}
Um vorhandene Ressourcen eines Servers zu erfahren, bietet \ac{CoAP} einen
Discovery Mechanismus.
Als Bedingung für das "`Entdecken"' eines Servers muss dieser lediglich auf dem
Standardport von \ac{CoAP}(5683) lauschen. Je nachdem, welche Protokolle \ac{CoAP}
zugrunde liegen, kann dann mittels Unicast, Multicast oder Broadcast der oder
die Server abefragt werden. Speziell für das Discover wurde der \ac{URI}-Path
\verb!/.well-known/core! reserviert. Dieser kann mittels einer \verb!GET!-Anfrage
abgerufen wurden. Die Antwort enthält dann ein
Linkformat (nach \cite{rfc6690}), in dem die angebotenen Ressourcen beschrieben
sind.
